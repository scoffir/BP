%%=============================================================================
%% Inleiding
%%=============================================================================

\chapter{\IfLanguageName{dutch}{Inleiding}{Introduction}}
\label{ch:inleiding}

Blockchain is een ICT-toepassing die de laatste jaren populair is geworden. Een van de meest bekende toepassingen van blockchain is bitcoin, maar er zijn nog andere toepassingen mogelijk buiten de financiële wereld. \autocite{Pilkington2016}.

\section{\IfLanguageName{dutch}{Probleemstelling}{Problem Statement}}
\label{sec:probleemstelling}

De coronapandemie van 2020 heeft getoond dat een snelle en veilige toegang van patiëntendossiers nodig is. Een van de mogelijkheden om dit op te lossen is het gebruik van een blockchaintoepassing. Deze technologische toepassing kan verregaande gevolgen hebben inzake de manier waarop onze samenleving georganiseerd wordt, maar er zijn verschillende mogelijkheden hoe dit systeem kan toegepast worden.

\section{\IfLanguageName{dutch}{Onderzoeksvraag}{Research question}}
\label{sec:onderzoeksvraag}

Het doel van deze bachelorproef is nagaan of blockchain verzamelpunt om patiëntendossiers op te slaan mogelijk is. Er zijn evenwel een paar vragen die beantwoord moeten worden en aspecten die onderzocht moeten worden.

\begin{itemize}
    \item Heeft de gezondheidsector kennis van de mogelijkheden van blockchain?
    \item Is er draagvlak voor dit systeem bij de gezondheidzorg?
    \item Is dit systeem in concordantie met Belgische privacywetgeving en Europese GDPR?
\end{itemize}

\section{\IfLanguageName{dutch}{Onderzoeksdoelstelling}{Research objective}}
\label{sec:onderzoeksdoelstelling}

Eerst gaat er een literatuurstudie bepaalt worden welke blockchaintoepassing het meeste kans op slagen heeft om in overeenstemming te zijn met de privacywetgeving en GDPR. 

Daarna zal er ook contact opgenomen worden met personen en organisaties die voor diverse overheidstoepassingen de mogelijkheden van Blockchain hebben onderzocht. Door middel van vragenlijsten gaat er gepeild worden naar de kennis en draagvlak van blockchain bij de gezondheidzorg. 

Aan de hand van deze gegevens gaat er een proof-of-concept ontwikkeld worden.

Daarna gaat er door middel van interviews bij advocaten, privacyspecialisten, rechters (en andere stakeholders die tijdens het maken van de bachelorproef geïdentificeerd worden) bepaalt worden in hoeverre de proof-of-concept in overeenstemming is met de privacywetgeving en GDPR. 

\section{\IfLanguageName{dutch}{Opzet van deze bachelorproef}{Structure of this bachelor thesis}}
\label{sec:opzet-bachelorproef}

% Het is gebruikelijk aan het einde van de inleiding een overzicht te
% geven van de opbouw van de rest van de tekst. Deze sectie bevat al een aanzet
% die je kan aanvullen/aanpassen in functie van je eigen tekst.

De rest van deze bachelorproef is als volgt opgebouwd:

In Hoofdstuk~\ref{ch:stand-van-zaken} wordt een overzicht gegeven van de stand van zaken binnen het onderzoeksdomein, op basis van een literatuurstudie.

In Hoofdstuk~\ref{ch:methodologie} wordt de methodologie toegelicht en worden de gebruikte onderzoekstechnieken besproken om een antwoord te kunnen formuleren op de onderzoeksvragen.

% TODO: Vul hier aan voor je eigen hoofstukken, één of twee zinnen per hoofdstuk

In Hoofdstuk~\ref{ch:conclusie}, tenslotte, wordt de conclusie gegeven en een antwoord geformuleerd op de onderzoeksvragen. Daarbij wordt ook een aanzet gegeven voor toekomstig onderzoek binnen dit domein.