%===============================================================================
% LaTeX sjabloon voor de bachelorproef toegepaste informatica aan HOGENT
% Meer info op https://github.com/HoGentTIN/bachproef-latex-sjabloon
%===============================================================================

\documentclass{bachproef-tin}

\usepackage{hogent-thesis-titlepage} % Titelpagina conform aan HOGENT huisstijl
\usepackage{titlesec}

%%---------- Documenteigenschappen ---------------------------------------------
% TODO: Vul dit aan met je eigen info:

% De titel van het rapport/bachelorproef
\title{Blockchain als verzamelpunt om patiëntendossiers op te slaan binnen de Belgische en Europese regelgeving}

% Je eigen naam
\author{Timothy Williame}

% De naam van je promotor (lector van de opleiding)
\promotor{Marc Asselberg}

% De naam van je co-promotor. Als je promotor ook je opdrachtgever is en je
% dus ook inhoudelijk begeleidt (en enkel dan!), mag je dit leeg laten.
\copromotor{Wim Van Renterghem}

% Indien je bachelorproef in opdracht van/in samenwerking met een bedrijf of
% externe organisatie geschreven is, geef je hier de naam. Zoniet laat je dit
% zoals het is.
\instelling{---}

% Academiejaar
\academiejaar{2020-2021}

% Examenperiode
%  - 1e semester = 1e examenperiode => 1
%  - 2e semester = 2e examenperiode => 2
%  - tweede zit  = 3e examenperiode => 3
\examenperiode{2}

%===============================================================================
% Inhoud document
%===============================================================================

\begin{document}

%---------- Taalselectie -------------------------------------------------------
% Als je je bachelorproef in het Engels schrijft, haal dan onderstaande regel
% uit commentaar. Let op: de tekst op de voorkaft blijft in het Nederlands, en
% dat is ook de bedoeling!

%\selectlanguage{english}

%---------- Titelblad ----------------------------------------------------------
\inserttitlepage

%---------- Samenvatting, voorwoord --------------------------------------------
\usechapterimagefalse
%%=============================================================================
%% Voorwoord
%%=============================================================================

\chapter*{\IfLanguageName{dutch}{Woord vooraf}{Preface}}
\label{ch:voorwoord}

%% TODO:
%% Het voorwoord is het enige deel van de bachelorproef waar je vanuit je
%% eigen standpunt (``ik-vorm'') mag schrijven. Je kan hier bv. motiveren
%% waarom jij het onderwerp wil bespreken.
%% Vergeet ook niet te bedanken wie je geholpen/gesteund/... heeft


%%=============================================================================
%% Samenvatting
%%=============================================================================

% TODO: De "abstract" of samenvatting is een kernachtige (~ 1 blz. voor een
% thesis) synthese van het document.
%
% Deze aspecten moeten zeker aan bod komen:
% - Context: waarom is dit werk belangrijk?
% - Nood: waarom moest dit onderzocht worden?
% - Taak: wat heb je precies gedaan?
% - Object: wat staat in dit document geschreven?
% - Resultaat: wat was het resultaat?
% - Conclusie: wat is/zijn de belangrijkste conclusie(s)?
% - Perspectief: blijven er nog vragen open die in de toekomst nog kunnen
%    onderzocht worden? Wat is een mogelijk vervolg voor jouw onderzoek?
%
% LET OP! Een samenvatting is GEEN voorwoord!

%%---------- Nederlandse samenvatting -----------------------------------------
%
% TODO: Als je je bachelorproef in het Engels schrijft, moet je eerst een
% Nederlandse samenvatting invoegen. Haal daarvoor onderstaande code uit
% commentaar.
% Wie zijn bachelorproef in het Nederlands schrijft, kan dit negeren, de inhoud
% wordt niet in het document ingevoegd.

\IfLanguageName{english}{%
\selectlanguage{dutch}
\chapter*{Samenvatting}
\selectlanguage{english}
}{}

%%---------- Samenvatting -----------------------------------------------------
% De samenvatting in de hoofdtaal van het document

\chapter*{\IfLanguageName{dutch}{Samenvatting}{Abstract}}


%---------- Inhoudstafel -------------------------------------------------------
\pagestyle{empty} % Geen hoofding
\tableofcontents  % Voeg de inhoudstafel toe
\cleardoublepage  % Zorg dat volgende hoofstuk op een oneven pagina begint
\pagestyle{fancy} % Zet hoofding opnieuw aan

%---------- Lijst figuren, afkortingen, ... ------------------------------------

% Indien gewenst kan je hier een lijst van figuren/tabellen opgeven. Geef in
% dat geval je figuren/tabellen altijd een korte beschrijving:
%
%  \caption[korte beschrijving]{uitgebreide beschrijving}
%
% De korte beschrijving wordt gebruikt voor deze lijst, de uitgebreide staat bij
% de figuur of tabel zelf.

\listoffigures
\listoftables

% Als je een lijst van afkortingen of termen wil toevoegen, dan hoort die
% hier thuis. Gebruik bijvoorbeeld de ``glossaries'' package.
% https://www.overleaf.com/learn/latex/Glossaries

%---------- Kern ---------------------------------------------------------------

% De eerste hoofdstukken van een bachelorproef zijn meestal een inleiding op
% het onderwerp, literatuurstudie en verantwoording methodologie.
% Aarzel niet om een meer beschrijvende titel aan deze hoofstukken te geven of
% om bijvoorbeeld de inleiding en/of stand van zaken over meerdere hoofdstukken
% te verspreiden!

%%=============================================================================
%% Inleiding
%%=============================================================================

\chapter{\IfLanguageName{dutch}{Inleiding}{Introduction}}
\label{ch:inleiding}

Blockchain is een ICT-toepassing die de laatste jaren populair is geworden. Een van de meest bekende toepassingen van blockchain is bitcoin, maar er zijn nog andere toepassingen mogelijk buiten de financiële wereld. \autocite{Pilkington2016}.

\section{\IfLanguageName{dutch}{Probleemstelling}{Problem Statement}}
\label{sec:probleemstelling}

De coronapandemie van 2020 heeft getoond dat een snelle en veilige toegang van patiëntendossiers nodig is. Een van de mogelijkheden om dit op te lossen is het gebruik van een blockchaintoepassing. Deze technologische toepassing kan verregaande gevolgen hebben inzake de manier waarop onze samenleving georganiseerd wordt, maar er zijn verschillende mogelijkheden hoe dit systeem kan toegepast worden.

\section{\IfLanguageName{dutch}{Onderzoeksvraag}{Research question}}
\label{sec:onderzoeksvraag}

Het doel van deze bachelorproef is nagaan of blockchain als centraal verzamelpunt om patiëntendossiers op te slaan mogelijk is. Er zijn evenwel een paar vragen die beantwoord moeten worden en aspecten die onderzocht moeten worden.

\begin{itemize}
    \item Heeft de gezondheidsector kennis van de mogelijkheden van blockchain?
    \item Is er draagvlak voor dit systeem bij de gezondheidzorg?
    \item Is dit systeem in concordantie met Belgische privacywetgeving en Europese GDPR?
\end{itemize}

\section{\IfLanguageName{dutch}{Onderzoeksdoelstelling}{Research objective}}
\label{sec:onderzoeksdoelstelling}

Wat is het beoogde resultaat van je bachelorproef? Wat zijn de criteria voor succes? Beschrijf die zo concreet mogelijk. Gaat het bv. om een proof-of-concept, een prototype, een verslag met aanbevelingen, een vergelijkende studie, enz.

\section{\IfLanguageName{dutch}{Opzet van deze bachelorproef}{Structure of this bachelor thesis}}
\label{sec:opzet-bachelorproef}

% Het is gebruikelijk aan het einde van de inleiding een overzicht te
% geven van de opbouw van de rest van de tekst. Deze sectie bevat al een aanzet
% die je kan aanvullen/aanpassen in functie van je eigen tekst.

De rest van deze bachelorproef is als volgt opgebouwd:

In Hoofdstuk~\ref{ch:stand-van-zaken} wordt een overzicht gegeven van de stand van zaken binnen het onderzoeksdomein, op basis van een literatuurstudie.

In Hoofdstuk~\ref{ch:methodologie} wordt de methodologie toegelicht en worden de gebruikte onderzoekstechnieken besproken om een antwoord te kunnen formuleren op de onderzoeksvragen.

% TODO: Vul hier aan voor je eigen hoofstukken, één of twee zinnen per hoofdstuk

In Hoofdstuk~\ref{ch:conclusie}, tenslotte, wordt de conclusie gegeven en een antwoord geformuleerd op de onderzoeksvragen. Daarbij wordt ook een aanzet gegeven voor toekomstig onderzoek binnen dit domein.
\chapter{\IfLanguageName{dutch}{Stand van zaken}{State of the art}}
\label{ch:stand-van-zaken}

% Tip: Begin elk hoofdstuk met een paragraaf inleiding die beschrijft hoe
% dit hoofdstuk past binnen het geheel van de bachelorproef. Geef in het
% bijzonder aan wat de link is met het vorige en volgende hoofdstuk.

% Pas na deze inleidende paragraaf komt de eerste sectiehoofding.

Dit hoofdstuk bevat je literatuurstudie. De inhoud gaat verder op de inleiding, maar zal het onderwerp van de bachelorproef *diepgaand* uitspitten. De bedoeling is dat de lezer na lezing van dit hoofdstuk helemaal op de hoogte is van de huidige stand van zaken (state-of-the-art) in het onderzoeksdomein. Iemand die niet vertrouwd is met het onderwerp, weet nu voldoende om de rest van het verhaal te kunnen volgen, zonder dat die er nog andere informatie moet over opzoeken \autocite{Pollefliet2011}.

Je verwijst bij elke bewering die je doet, vakterm die je introduceert, enz. naar je bronnen. In \LaTeX{} kan dat met het commando \texttt{$\backslash${textcite\{\}}} of \texttt{$\backslash${autocite\{\}}}. Als argument van het commando geef je de ``sleutel'' van een ``record'' in een bibliografische databank in het Bib\LaTeX{}-formaat (een tekstbestand). Als je expliciet naar de auteur verwijst in de zin, gebruik je \texttt{$\backslash${}textcite\{\}}.
Soms wil je de auteur niet expliciet vernoemen, dan gebruik je \texttt{$\backslash${}autocite\{\}}. In de volgende paragraaf een voorbeeld van elk.

\textcite{Knuth1998} schreef een van de standaardwerken over sorteer- en zoekalgoritmen. Experten zijn het erover eens dat cloud computing een interessante opportuniteit vormen, zowel voor gebruikers als voor dienstverleners op vlak van informatietechnologie~\autocite{Creeger2009}.

\lipsum[7-20]

%%=============================================================================
%% Methodologie
%%=============================================================================

\chapter{\IfLanguageName{dutch}{Methodologie}{Methodology}}
\label{ch:methodologie}

%% TODO: Hoe ben je te werk gegaan? Verdeel je onderzoek in grote fasen, en
%% licht in elke fase toe welke stappen je gevolgd hebt. Verantwoord waarom je
%% op deze manier te werk gegaan bent. Je moet kunnen aantonen dat je de best
%% mogelijke manier toegepast hebt om een antwoord te vinden op de
%% onderzoeksvraag.

\lipsum[21-25]



% Voeg hier je eigen hoofdstukken toe die de ``corpus'' van je bachelorproef
% vormen. De structuur en titels hangen af van je eigen onderzoek. Je kan bv.
% elke fase in je onderzoek in een apart hoofdstuk bespreken.

%\input{...}
%\input{...}
%...

%%=============================================================================
%% Conclusie
%%=============================================================================

\chapter{Conclusie}
\label{ch:conclusie}

% TODO: Trek een duidelijke conclusie, in de vorm van een antwoord op de
% onderzoeksvra(a)g(en). Wat was jouw bijdrage aan het onderzoeksdomein en
% hoe biedt dit meerwaarde aan het vakgebied/doelgroep? 
% Reflecteer kritisch over het resultaat. In Engelse teksten wordt deze sectie
% ``Discussion'' genoemd. Had je deze uitkomst verwacht? Zijn er zaken die nog
% niet duidelijk zijn?
% Heeft het onderzoek geleid tot nieuwe vragen die uitnodigen tot verder 
%onderzoek?



%%=============================================================================
%% Bijlagen
%%=============================================================================

\appendix
\renewcommand{\chaptername}{Appendix}

%%---------- Onderzoeksvoorstel -----------------------------------------------

\chapter{Onderzoeksvoorstel}

Het onderwerp van deze bachelorproef is gebaseerd op een onderzoeksvoorstel dat vooraf werd beoordeeld door de promotor. Dat voorstel is opgenomen in deze bijlage.

% Verwijzing naar het bestand met de inhoud van het onderzoeksvoorstel
%---------- Inleiding ---------------------------------------------------------

\section{Introductie} % The \section*{} command stops section numbering
\label{sec:introductie}

Blockchain is een ICT-toepassing die de laatste jaren populair is geworden. Een van de meest bekende toepassingen van blockchain is bitcoin, maar er zijn nog andere toepassingen mogelijk buiten de financiële wereld. \autocite{Pilkington2016}.

De coronapandemie van 2020 heeft getoond dat een snelle en veilige toegang van patiëntendossiers nodig is. Een van de mogelijkheden om dit op te lossen is het gebruik van een blockchaintoepassing. Deze technologische toepassing kan verregaande gevolgen hebben inzake de manier waarop onze samenleving georganiseerd wordt, maar er zijn verschillende mogelijkheden hoe dit systeem kan toegepast worden. 

Het doel van deze bachelorproef is nagaan of blockchain als centraal verzamelpunt om patiëntendossiers op te slaan mogelijk is. Er zijn evenwel een paar vragen die beantwoord moeten worden en aspecten die onderzocht moeten worden.

\begin{itemize}
    \item Heeft de gezondheidsector kennis van de mogelijkheden van blockchain?
    \item Is er draagvlak voor dit systeem bij de gezondheidzorg?
    \item Is dit systeem in concordantie met Belgische privacywetgeving en Europese GDPR?
\end{itemize}


%---------- Stand van zaken ---------------------------------------------------

\section{Stand van zaken}
\label{sec:stand van zaken}

Er bestaat op dit moment geen blockchaintoepassing in de gezondheidzorg die volledig concordant is met GDPR, maar er kunnen wel specifieke use cases zijn in overeenstemming zijn met GDPR \autocite{Hasselgren2020}. Het onderzoek van Hasselgren, Wan, Horn, Kralevska, Gligorski en Faxvaag \autocite{Hasselgren2020} vergeleek 4 blockchaintoepassingen met de GDPR en besloot dat sommige onderdelen van blockchain GDPR versterken, maar op andere punten botst met de regels.

Het grootste probleem bij de combinatie GDPR-blockchain is het recht om vergeten te worden vanwege de eigenschap dat een blockchain onveranderlijk is \autocite{Pilkington2016}. Volgens de studie van Mirchandani \autocite{Mirchandani2019} ligt deze moeilijkheid in de slechte definitie van het woord ``Erasure'' in artikel 17. Dit feit wordt ook benadrukt in de studie van Finck \autocite{Finck2019}. Deze studie herhaalt ook dat compatibiliteit van blockchaintoepassingen met de GDPR case per case moet beoordeelt worden.

In deze bachelorproef gaan we kijken of er een blockchaintoepassing kan gecreëerd worden waarbij data met behulp van merkle hash trees versleuteld wordt \autocite{Niaz2015}. 

%---------- Methodologie ------------------------------------------------------
\section{Methodologie}
\label{sec:methodologie}

Eerst gaat er een literatuurstudie bepaalt worden welke blockchaintoepassing het meeste kans op slagen heeft om in overeenstemming te zijn met de privacywetgeving en GDPR. Daarnaast zal er ook contact opgenomen worden met personen en organisaties die voor diverse overheidstoepassingen de mogelijkheden van Blockchain hebben onderzocht. Door middel van vragenlijsten gaat er gepeild worden naar de kennis en draagvlak van blockchain bij de gezondheidzorg. Daarnaast gaat er door middel van interviews bij advocaten, privacyspecialisten, rechters (en andere stakeholders die tijdens het maken van de bachelorproef geïdentificeerd worden) bepaalt worden in hoeverre de blockchaintoepassing in overeenstemming is met de privacywetgeving en GDPR. Uiteindelijk gaat er een proof-of-concept ontwikkeld worden die getoetst gaat worden aan de verworven inzichten.

%---------- Verwachte resultaten ----------------------------------------------
\section{Verwachte resultaten}
\label{sec:verwachte_resultaten}

Er wordt verwacht dat uit de literatuurstudie zal blijken dat een private en permissioned blockchain met een merkle hash tree de meeste kans gaat hebben om in overeenstemming te zijn met de GDPR en privacywetgeving. Uit de bevraging met de gezondheidzorg wordt er verwacht dat de kennis over blockchain niet erg groot is, maar dat er wel veel draagvlak is voor een blockchaintoepassing vanwege de basiseigenschappen van een blockchain. Het gebruik van merkle hash trees zal voor de Belgische privacywetgeving in orde zijn, maar er wordt verwacht dat er geen eenduidig antwoord zal zijn op gebied van de overeenstemming met de GDPR. Er zal een proof-of-concept kunnen otwikkeld worden, maar deze zal wel voldoen aan de Belgische privacywetgeving niet voldoen aan de GDPR

%---------- Verwachte conclusies ----------------------------------------------
\section{Verwachte conclusies}
\label{sec:verwachte_conclusies}

De verwachte conclusie is dat de creatie van deze blockchaintoepassing mogelijk is en dat er een draagvlak voor is binnen de gezondheidszorg. Deze toepassing zal in een proof-of-concept ontwikkeld worden, maar deze toepassing zal na analyse en evaluatie evenwel niet gebruikt kunnen worden wegens de onduidelijke definitie van het woord ``erasure'' in artikel 17 van de GDPR. Dit probleem zal elke blockchaintoepassing die persoonlijke data gebruikt in de weg staan tenzij deze een duidelijkere omschrijving krijgt.



%%---------- Andere bijlagen --------------------------------------------------
% TODO: Voeg hier eventuele andere bijlagen toe
%\input{...}

%%---------- Referentielijst --------------------------------------------------

\printbibliography[heading=bibintoc]


\end{document}
