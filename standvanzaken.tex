\chapter{\IfLanguageName{dutch}{Stand van zaken}{State of the art}}
\label{ch:stand-van-zaken}

Zoals reeds in het vorige hoofdstuk aangegeven zal dit onderzoek zich richten op blockchain en de mogelijkheid om deze technologie te gebruiken om patiëntendossiers op te slaan. We bestuderen de blockchaintechnologie, wat deze inhoudt en hoe we dit gaan gebruiken. We gaan ook de verschillende wetgevingen zoals onder andere de Belgische privacywetgeving, de wet betreffende de verwerking van persoonsgegevens en de Europese GDPR.
% Tip: Begin elk hoofdstuk met een paragraaf inleiding die beschrijft hoe
% dit hoofdstuk past binnen het geheel van de bachelorproef. Geef in het
% bijzonder aan wat de link is met het vorige en volgende hoofdstuk.
% Pas na deze inleidende paragraaf komt de eerste sectiehoofding.
\section{Blockchain}
\label{ch:blockchain}

Blockchain is een fraudebestendig ``ledger'' of grootboek die op een gedistribueerde manier geïmplementeerd zijn \autocite{Yaga2018}. Blockchain is vooral bekend van zijn gebruik in de financiële wereld waar het de technologie achter onder andere Bitcoin is \autocite{Nofer2017}. 

\subsection{Blockchain categorieën}

\paragraph{Permissionless}

\textcite{Yaga2018} omschrijft permisionless blockchain als een gedecentraliseerde ``ledger'' waarin iedereen data kan publiceren zonder toestemming van een autoriteit. Bitcoin is een goed voorbeeld van deze categorie van blockchain. Doordat iedereen de inhoud van de blocks kan schrijven en lezen is permisionless blockchain niet geschikt voor privacygevoelige zaken zoals patiëntendossiers.


\paragraph{Permissioned}

In hetzelfde onderzoeksrapport omschrijft \textcite{Yaga2018} permissioned blockchain als een ``ledger'' waarin het schrijven en lezen van nieuwe blocks geauthoriseerd moeten worden door een authoriteit zoals een overheid, bedrijf, ea. Deze autoriteit kan gecentraliseerd of gedecentraliseerd zijn.

De bescherming van de schrijf -en leesrechten zorgen ervoor dat deze categorie van blockchain het best geschikt is om vertrouwelijke en persoonlijke data zoals patiëntendossiers te behandelen.

