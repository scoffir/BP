\chapter{\IfLanguageName{dutch}{Stand van zaken}{State of the art}}
\label{ch:stand-van-zaken}

Zoals reeds in het vorige hoofdstuk aangegeven zal dit onderzoek zich richten op blockchain en de mogelijkheid om deze technologie te gebruiken om patiëntendossiers op te slaan. We bestuderen de blockchaintechnologie, wat deze inhoudt en hoe we dit gaan gebruiken. We gaan ook de verschillende wetgevingen zoals onder andere de Belgische privacywetgeving, de wet betreffende de verwerking van persoonsgegevens en de Europese GDPR.

\section{Blockchain}
\label{ch:blockchain} 

\subsection{Inleiding}

Blockchain wordt gedefinieerd als een fraudebestendig ``ledger'' of grootboek die op een gedistribueerde manier geïmplementeerd is en meestal zonder een centrale autoriteit. Op zijn eenvoudigste niveau laat gebruikers toe transacties uit te voeren in een gemeenschappelijk grootboek. Bij een normale werking van het netwerk kunnen deze transacties niet meer veranderd worden nadat ze gepubliceerd zijn. Blockchain is vooral bekend van zijn gebruik in de financiële wereld waar het de technologie achter onder andere Bitcoin is.\autocite{Yaga2018}

\subsection{Blockchain eigenschappen}

\subsection{Blockchain categorieën}

\subsubsection{Permissionless}

\textcite{Yaga2018} omschrijft permisionless blockchain als een gedecentraliseerde ``ledger'' waarin iedereen data kan publiceren zonder toestemming van een autoriteit. Bitcoin is een goed voorbeeld van deze categorie van blockchain.

Er zijn wel een paar nadelen bij het gebruik van permissionless blockchain zoals beschreven in het artikel van \textcite{Liu}:
\begin{itemize}
    \item De snelheidslimiet bij het verwerken van grote hoeveelheden transacties werkt het oproepen van een patiëntendossier tegen als de blockchain grootschalig wordt.
    \item Een groter probleem is echter de privacy bescherming. In andere businessdomeinen hebben eigenaars bezorgdheden over de veiligheid van vertrouwelijke informatie.
\end{itemize}


Doordat iedereen de inhoud van de blocks kan schrijven en lezen is permisionless blockchain niet geschikt voor privacygevoelige zaken zoals patiëntendossiers.


\paragraph{Permissioned}

In hetzelfde onderzoeksrapport omschrijft \textcite{Yaga2018} permissioned blockchain als een ``ledger'' waarin het schrijven en lezen van nieuwe blocks geauthoriseerd moeten worden door een authoriteit zoals een overheid, bedrijf, ea. Deze autoriteit kan gecentraliseerd of gedecentraliseerd zijn.

De bescherming van de schrijf -en leesrechten zorgen ervoor dat deze categorie van blockchain het best geschikt is om vertrouwelijke en persoonlijke data zoals patiëntendossiers te behandelen.

Hoewel blockchain in zijn basis een trustless systeem is dit niet noodzakelijk. In sommige gevallen kan het zelfs beter zijn om een gecentraliseerde autoriteit te hebben die bepaalt wie er schrijf en/of leesrechten heeft binnen de blockchain. Deze autoriteit moet wel de reputatie hebben betrouwenswaardig te zijn van nature op het gebied van gegevensbescherming \autocite{Bacon2018}. 

Men kan bevoorbeeld van overheden of zorgnetten verwachten dat ze de gegevens beschermen en aanvullen, maar niet vervalsen aangezien ze dit nu al doen met de huidige patiëntendossiers buiten een blockchainomgeving.


\section{Belgische wetgeving}


\section{GDPR}