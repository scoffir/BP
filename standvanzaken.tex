\chapter{\IfLanguageName{dutch}{Stand van zaken}{State of the art}}
\label{ch:stand-van-zaken}

Zoals reeds in het vorige hoofdstuk aangegeven zal dit onderzoek zich richten op blockchain en de mogelijkheid om deze technologie te gebruiken om patiëntendossiers op te slaan. We bestuderen de blockchaintechnologie, wat deze inhoudt en hoe we dit gaan gebruiken. We gaan ook de verschillende wetgevingen zoals onder andere de Belgische privacywetgeving, de wet betreffende de verwerking van persoonsgegevens en de Europese GDPR.
% Tip: Begin elk hoofdstuk met een paragraaf inleiding die beschrijft hoe
% dit hoofdstuk past binnen het geheel van de bachelorproef. Geef in het
% bijzonder aan wat de link is met het vorige en volgende hoofdstuk.
% Pas na deze inleidende paragraaf komt de eerste sectiehoofding.
\section{Blockchain}
\label{ch:blockchain}

Blockchain is een fraudebestendig ``ledger'' of grootboek die op een gedistribueerde manier geïmplementeerd zijn en dus geen centraal verzamelpunt hebben en die ook geen centale autoriteit zoals een bedrijf of overheid hebben \autocite{Yaga2018}. Blockchain is vooral bekend van zijn gebruik in de financiële wereld waar het de technologie achter onder andere Bitcoin is \autocite{Nofer2017}. 

Er bestaat op dit moment geen blockchaintoepassing in de gezondheidzorg die volledig concordant is met GDPR, maar er kunnen wel specifieke use cases zijn in overeenstemming zijn met GDPR \autocite{Hasselgren2020}. Het onderzoek van \textcite{Hasselgren2020} vergeleek 4 blockchaintoepassingen met de GDPR en besloot dat sommige onderdelen van blockchain GDPR versterken, maar op andere punten botst met de regels.

Het grootste probleem bij de combinatie GDPR-blockchain is het recht om vergeten te worden vanwege de eigenschap dat een blockchain onveranderlijk is \autocite{Pilkington2016}. Volgens de studie van Mirchandani \autocite{Mirchandani2019} ligt deze moeilijkheid in de slechte definitie van het woord ``Erasure'' in artikel 17. Dit feit wordt ook benadrukt in de studie van Finck \autocite{Finck2019}. Deze studie herhaalt ook dat compatibiliteit van blockchaintoepassingen met de GDPR case per case moet beoordeelt worden.

In deze bachelorproef gaan we kijken of er een blockchaintoepassing kan gecreëerd worden waarbij data met behulp van merkle hash trees versleuteld wordt \autocite{Niaz2015}.




