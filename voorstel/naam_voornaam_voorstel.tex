%==============================================================================
% Sjabloon onderzoeksvoorstel bachelorproef
%==============================================================================
% Gebaseerd op LaTeX-sjabloon ‘Stylish Article’ (zie voorstel.cls)
% Auteur: Jens Buysse, Bert Van Vreckem
%
% Compileren in TeXstudio:
%
% - Zorg dat Biber de bibliografie compileert (en niet Biblatex)
%   Options > Configure > Build > Default Bibliography Tool: "txs:///biber"
% - F5 om te compileren en het resultaat te bekijken.
% - Als de bibliografie niet zichtbaar is, probeer dan F5 - F8 - F5
%   Met F8 compileer je de bibliografie apart.
%
% Als je JabRef gebruikt voor het bijhouden van de bibliografie, zorg dan
% dat je in ``biblatex''-modus opslaat: File > Switch to BibLaTeX mode.

\documentclass{voorstel}

\usepackage{lipsum}

%------------------------------------------------------------------------------
% Metadata over het voorstel
%------------------------------------------------------------------------------

%---------- Titel & auteur ----------------------------------------------------

% TODO: geef werktitel van je eigen voorstel op
\PaperTitle{Titel voorstel}
\PaperType{Onderzoeksvoorstel Bachelorproef 2019-2020} % Type document

% TODO: vul je eigen naam in als auteur, geef ook je emailadres mee!
\Authors{Steven Stevens\textsuperscript{1}} % Authors
\CoPromotor{Piet Pieters\textsuperscript{2} (Bedrijfsnaam)}
\affiliation{\textbf{Contact:}
  \textsuperscript{1} \href{mailto:steven.stevens.u1234@student.hogent.be}{steven.stevens.u1234@student.hogent.be};
  \textsuperscript{2} \href{mailto:piet.pieters@acme.be}{piet.pieters@acme.be};
}

%---------- Abstract ----------------------------------------------------------

\Abstract{Hier schrijf je de samenvatting van je voorstel, als een doorlopende tekst van één paragraaf. Wat hier zeker in moet vermeld worden: \textbf{Context} (Waarom is dit werk belangrijk?); \textbf{Nood} (Waarom moet dit onderzocht worden?); \textbf{Taak} (Wat ga je (ongeveer) doen?); \textbf{Object} (Wat staat in dit document geschreven?); \textbf{Resultaat} (Wat verwacht je van je onderzoek?); \textbf{Conclusie} (Wat verwacht je van van de conclusies?); \textbf{Perspectief} (Wat zegt de toekomst voor dit werk?).

Bij de sleutelwoorden geef je het onderzoeksdomein, samen met andere sleutelwoorden die je werk beschrijven.

Vergeet ook niet je co-promotor op te geven.
}

%---------- Onderzoeksdomein en sleutelwoorden --------------------------------
% TODO: Sleutelwoorden:
%
% Het eerste sleutelwoord beschrijft het onderzoeksdomein. Je kan kiezen uit
% deze lijst:
%
% - Mobiele applicatieontwikkeling
% - Webapplicatieontwikkeling
% - Applicatieontwikkeling (andere)
% - Systeembeheer
% - Netwerkbeheer
% - Mainframe
% - E-business
% - Databanken en big data
% - Machineleertechnieken en kunstmatige intelligentie
% - Andere (specifieer)
%
% De andere sleutelwoorden zijn vrij te kiezen

\Keywords{Onderzoeksdomein. Keyword1 --- Keyword2 --- Keyword3} % Keywords
\newcommand{\keywordname}{Sleutelwoorden} % Defines the keywords heading name

%---------- Titel, inhoud -----------------------------------------------------

\begin{document}

\flushbottom % Makes all text pages the same height
\maketitle % Print the title and abstract box
\tableofcontents % Print the contents section
\thispagestyle{empty} % Removes page numbering from the first page

%------------------------------------------------------------------------------
% Hoofdtekst
%------------------------------------------------------------------------------

% De hoofdtekst van het voorstel zit in een apart bestand, zodat het makkelijk
% kan opgenomen worden in de bijlagen van de bachelorproef zelf.
%---------- Inleiding ---------------------------------------------------------

\section{Introductie} % The \section*{} command stops section numbering
\label{sec:introductie}

Blockchain is een ICT-toepassing die de laatste jaren populair is geworden. Een van de meest bekende toepassingen van blockchain is bitcoin, maar er zijn nog andere toepassingen mogelijk buiten de financiële wereld. \autocite{Pilkington2016}.

De coronapandemie van 2020 heeft getoond dat een snelle en veilige toegang van patiëntendossiers nodig is. Een van de mogelijkheden om dit op te lossen is het gebruik van een blockchaintoepassing. Deze technologische toepassing kan verregaande gevolgen hebben inzake de manier waarop onze samenleving georganiseerd wordt, maar er zijn verschillende mogelijkheden hoe dit systeem kan toegepast worden. 

Het doel van deze bachelorproef is nagaan of blockchain als centraal verzamelpunt om patiëntendossiers op te slaan mogelijk is. Er zijn evenwel een paar vragen die beantwoord moeten worden en aspecten die onderzocht moeten worden.

\begin{itemize}
    \item Heeft de gezondheidsector kennis van de mogelijkheden van blockchain?
    \item Is er draagvlak voor dit systeem bij de gezondheidzorg?
    \item Is dit systeem in concordantie met Belgische privacywetgeving en Europese GDPR?
\end{itemize}


%---------- Stand van zaken ---------------------------------------------------

\section{Stand van zaken}
\label{sec:stand van zaken}

Er bestaat op dit moment geen blockchaintoepassing in de gezondheidzorg die volledig concordant is met GDPR, maar er kunnen wel specifieke use cases zijn in overeenstemming zijn met GDPR \autocite{Hasselgren2020}. Het onderzoek van Hasselgren, Wan, Horn, Kralevska, Gligorski en Faxvaag \autocite{Hasselgren2020} vergeleek 4 blockchaintoepassingen met de GDPR en besloot dat sommige onderdelen van blockchain GDPR versterken, maar op andere punten botst met de regels.

Het grootste probleem bij de combinatie GDPR-blockchain is het recht om vergeten te worden vanwege de eigenschap dat een blockchain onveranderlijk is \autocite{Pilkington2016}. Volgens de studie van Mirchandani \autocite{Mirchandani2019} ligt deze moeilijkheid in de slechte definitie van het woord ``Erasure'' in artikel 17. Dit feit wordt ook benadrukt in de studie van Finck \autocite{Finck2019}. Deze studie herhaalt ook dat compatibiliteit van blockchaintoepassingen met de GDPR case per case moet beoordeelt worden.

In deze bachelorproef gaan we kijken of er een blockchaintoepassing kan gecreëerd worden waarbij data met behulp van merkle hash trees versleuteld wordt \autocite{Niaz2015}. 

%---------- Methodologie ------------------------------------------------------
\section{Methodologie}
\label{sec:methodologie}

Eerst gaat er een literatuurstudie bepaalt worden welke blockchaintoepassing het meeste kans op slagen heeft om in overeenstemming te zijn met de privacywetgeving en GDPR. Daarnaast zal er ook contact opgenomen worden met personen en organisaties die voor diverse overheidstoepassingen de mogelijkheden van Blockchain hebben onderzocht. Door middel van vragenlijsten gaat er gepeild worden naar de kennis en draagvlak van blockchain bij de gezondheidzorg. Daarnaast gaat er door middel van interviews bij advocaten, privacyspecialisten, rechters (en andere stakeholders die tijdens het maken van de bachelorproef geïdentificeerd worden) bepaalt worden in hoeverre de blockchaintoepassing in overeenstemming is met de privacywetgeving en GDPR. Uiteindelijk gaat er een proof-of-concept ontwikkeld worden die getoetst gaat worden aan de verworven inzichten.

%---------- Verwachte resultaten ----------------------------------------------
\section{Verwachte resultaten}
\label{sec:verwachte_resultaten}

Er wordt verwacht dat uit de literatuurstudie zal blijken dat een private en permissioned blockchain met een merkle hash tree de meeste kans gaat hebben om in overeenstemming te zijn met de GDPR en privacywetgeving. Uit de bevraging met de gezondheidzorg wordt er verwacht dat de kennis over blockchain niet erg groot is, maar dat er wel veel draagvlak is voor een blockchaintoepassing vanwege de basiseigenschappen van een blockchain. Het gebruik van merkle hash trees zal voor de Belgische privacywetgeving in orde zijn, maar er wordt verwacht dat er geen eenduidig antwoord zal zijn op gebied van de overeenstemming met de GDPR. Er zal een proof-of-concept kunnen otwikkeld worden, maar deze zal wel voldoen aan de Belgische privacywetgeving niet voldoen aan de GDPR

%---------- Verwachte conclusies ----------------------------------------------
\section{Verwachte conclusies}
\label{sec:verwachte_conclusies}

De verwachte conclusie is dat de creatie van deze blockchaintoepassing mogelijk is en dat er een draagvlak voor is binnen de gezondheidszorg. Deze toepassing zal in een proof-of-concept ontwikkeld worden, maar deze toepassing zal na analyse en evaluatie evenwel niet gebruikt kunnen worden wegens de onduidelijke definitie van het woord ``erasure'' in artikel 17 van de GDPR. Dit probleem zal elke blockchaintoepassing die persoonlijke data gebruikt in de weg staan tenzij deze een duidelijkere omschrijving krijgt.



%------------------------------------------------------------------------------
% Referentielijst
%------------------------------------------------------------------------------
% TODO: de gerefereerde werken moeten in BibTeX-bestand ``voorstel.bib''
% voorkomen. Gebruik JabRef om je bibliografie bij te houden en vergeet niet
% om compatibiliteit met Biber/BibLaTeX aan te zetten (File > Switch to
% BibLaTeX mode)

\phantomsection
\printbibliography[heading=bibintoc]

\end{document}
